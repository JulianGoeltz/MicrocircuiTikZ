\begin{tikzpicture}[
    % font={\sffamily \small},
    scale=1.,
    >=latex,
    transform shape,
]
    \useasboundingbox (0,0) rectangle (4.7, 4.4);

    \def\numInputs{4}
    \def\numHidden{5}
    \def\numLabel{4}

    \def\layerSeparation{1.2}
    \def\neuronSeparation{0.8}
    \def\xshiftInput{0.5 * \neuronSeparation}
    \def\xshiftHidden{0.0 * \neuronSeparation}
    \def\xshiftLabel{0.5 * \neuronSeparation}

    \def\yOffset{0.5}
    % This is the same as writing \foreach \name / \y in {1/1,2/2,3/3,4/4}
    \foreach \name / \i in {1,...,\numInputs}
        \pyramidalNeuron{\xshiftInput + \i * \neuronSeparation}{\yOffset}{}{input-\name}{0.8}{block};

    \pgfmathsetmacro{\yOffset}{\yOffset + \layerSeparation}
    \foreach \name / \i in {1,...,\numHidden}
        \pyramidalNeuron{\xshiftHidden + \i * \neuronSeparation}{\yOffset}{}{hidden-\name}{0.8}{};

    \pgfmathsetmacro{\yOffset}{\yOffset + \layerSeparation}
    \foreach \name / \i in {1,...,\numLabel}
        \pyramidalNeuron{\xshiftLabel + \i * \neuronSeparation}{\yOffset}{}{label-\name}{0.8}{};



    \foreach \name / \i in {1,...,\numLabel}{
        \draw (2.25, \yOffset + 0.9) 
            .. controls ([xshift=0.2cm]label-\name-apicalTipR)
            % .. controls (\xshiftLabel + \i * \neuronSeparation + 2 * \constApicalBrnX, \yOffset - 0.25 * \layerSeparation)
            .. (label-\name-apicalR)
        ;

    }
    \node[ anchor=south] at (2.25, \yOffset + 0.95) {$1-p(u)$};



\end{tikzpicture}

